\documentclass[12pt,a4paper]{article}

% Packages
\usepackage[utf8]{inputenc}
\usepackage[T1]{fontenc}
\usepackage{amsmath}
\usepackage{amsfonts}
\usepackage{amssymb}
\usepackage{graphicx}
\usepackage{hyperref}
\usepackage{geometry}
\usepackage{multicol}
\geometry{a4paper,landscape,left=0.2cm,right=0.2cm,top=0.2cm,bottom=0.2cm}
\setlength{\columnsep}{0.1cm}
\pagestyle{empty}

% Title information
\title{Sample LaTeX Document}
\author{Your Name}
\date{\today}

\begin{document}

\begin{multicols*}{5}

\section{Introduction}

This is a sample LaTeX document. LaTeX is a high-quality typesetting system that is particularly useful for technical and scientific documents.

\section{Basic Formatting}

\subsection{Text Formatting}

You can use \textbf{bold text}, \textit{italic text}, and \underline{underlined text}. You can also use \texttt{monospace text} for code.

\subsection{Lists}

Here's an unordered list:
\begin{itemize}
    \item First item
    \item Second item
    \item Third item
\end{itemize}

And here's an ordered list:
\begin{enumerate}
    \item First numbered item
    \item Second numbered item
    \item Third numbered item
\end{enumerate}

\section{Mathematics}

LaTeX excels at typesetting mathematics. Here are some examples:

Inline math: The quadratic formula is $x = \frac{-b \pm \sqrt{b^2 - 4ac}}{2a}$.

Display math:
\begin{equation}
    E = mc^2
\end{equation}

A more complex equation:
\begin{equation}
    \int_{-\infty}^{\infty} e^{-x^2} dx = \sqrt{\pi}
\end{equation}

\section{Tables}

Here's a simple table:

\begin{table}[h]
\centering
\begin{tabular}{|c|c|c|}
\hline
\textbf{Column 1} & \textbf{Column 2} & \textbf{Column 3} \\
\hline
Row 1, Col 1 & Row 1, Col 2 & Row 1, Col 3 \\
Row 2, Col 1 & Row 2, Col 2 & Row 2, Col 3 \\
\hline
\end{tabular}
\caption{A sample table}
\label{tab:sample}
\end{table}

\section{References}

You can reference sections, equations, and tables. For example, see Table~\ref{tab:sample} or Equation~\ref{eq:quadratic}.

\begin{equation}
    \label{eq:quadratic}
    ax^2 + bx + c = 0
\end{equation}

\section{Conclusion}

This sample document demonstrates many of the basic features of LaTeX. For more information, visit \url{https://www.latex-project.org/}.

\end{multicols*}

\end{document}

